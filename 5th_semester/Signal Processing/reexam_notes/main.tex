\documentclass{article}
%packages
\usepackage[a4paper, total={6in, 8in}]{geometry}
\usepackage{float}
\title{Signal Processing Notes}
\author{Thomas Møller Jensen}
\begin{document}
\maketitle
\section*{Pre course: Complex numbers}
\section*{Topic 1: Introduction to signals}
A discrete signal is generally defined as a sequence of numbers \- A set in other words.\\
There are multiple different basic sequences that you can consider, these will be shown below:
\begin{table}[H]
	\centering
	\begin{tabular}{|c|c|}
		\hline
		\textbf{Name} & \textbf{Definition} \\\hline
		sequence & $x[n]=\sum_{k=-\infty}^{\infty} x[k]\delta [n-k]$ \\\hline
		unit sample sequence & $\delta [n]=\{^{0, n\neq 0,}_{0, n = 0.}$ \\\hline
		unit step sequence & $u[n]=\{^{1, n \leq 0,}_{0, n<0.}$ \\\hline
	\end{tabular}
	
\end{table}

\section*{Topic 2: Convolution}
Convolution is usually denoted by the asterisk (*) 

the definition for this notation:

\begin{equation}
y[n]=x[n]*h[n]
\end{equation}

\paragraph{LTI}

the sequence y[n] can be found by convoluting h[n] and x[n]

\begin{equation}
y[n]=\sum_{k=-\infty}^{\infty}x[k]h[n-k]
\end{equation}
\subsection*{Exercises}





\section*{Topic 3: z-transform}
$X(z)=\sum_{n=-\infty}^{\infty}x[n]z^{-n}$

\subsection*{Exercises}
Determine the z-transform including the ROC, for each of the following sequences

\begin{enum}
	\item $(\frac12)^nu[n]$
	
\end{enum}
\section*{Topic 4: Sampling}
\section*{Topic 5: Analog Filters}
\section*{Topic 6: Digial IIR Filters}
\subsection*{The Impulse Invariant Method}
\section*{Topic 7: Design of IIR Digital Filters}
\section*{Topic 8: Digital FIR Filters}
\subsection*{Linear Phase}
\subsection*{The Window Method}
\section*{Topic 9: Frequency Analysis of LTI Systems}
\section*{Topic 10: Realization Structures for Digital Filters}
\section*{Topic 11\-13: The Discrete Fourier Transform}
\section*{Topic 14: The Short Time Fourier Transform (STFT)}

\section*{2023 exam set}
\subsection{10\% Spectral Estimation}
A continuous-time signal $x_c(t)$ has the following spectral representation,
\begin{equation}
	|X_c(\Omega)|=\{
\end{equation}

\end{document}
