\section*{Extra Lecture \#1: probability summary}
We're working with events

The total set of all results is called the `Sample Set' S

an event A could be any permutation of values in S

\subsection*{Conditional Probability}
The notation is as follows `$P(A|B)$` read as `Probability of A given B'

This means that we assume that B is already happened and we determine A based on the outcome of B

Example: $P(\text{weekend | yest.\ friday})$, then the probability of it being weekend given that it was friday yesterday

$P(\text{weekend})$ will be 2/7 given the set of all days in a week

\subsection*{independance}
A and B are independant <=>

$P(AB) = P(A)P(B)$

because $A \cup B = AB$

it should also be true that $P(ABC) = P(A)P(B)P(C)$ if $P(AC) = P(A)P(C)$ should also be true

we don't make venn diagrams for these as it doesn't really make sense

as for conditional probability, if A and B are independant, then $P(A|B) = P(A)$

\subsection*{Mutual exclusion}
If A and B is mutually exclusive that means no values in A is in B

in terms of probability, that means that if we want A given B ($P(A|B)$) then that will always equal 0

\subsection*{Combinatorics}
k persons in a room

probability that at least 2 persons have a birthday on the same day

$P(\text{at least}2) = P(2)+P(3)+P(4) \dots P(k)$

$= 1 - P(\text{noone have a birthday on the same day})$

$P(\text{noone}) = \frac{\text{\# of options when noone has the same birthday}}{\text{\# of all possible options}}$

this example is called the birthday paradox

