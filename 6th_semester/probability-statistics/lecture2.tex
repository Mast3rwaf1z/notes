\section*{Topic 2: Random Variables}
A random variable is usually a real value which outcome is determined by the outcome of a experiment

\paragraph{Example:}

Consider the sample space $S=\{00000, 00001, \ldots, 11111\}$

here the random variable will be called $X$, and X has the value of the sum of the value of X. so lets say the experiment has the outcome $10101$, the value of $X=3$

in random variables, they will also usually have a range, the range is denoted as $R_X$ for the variable $X$, the range of a random variable is a set of all possible values of the variable



\subsection*{Exercises 3.2}
Find the range for each of the following random variables.

	\begin{enumerate}
		\item I toss a coin 100 times. Let X be the number of heads i observe.
	
		I observe that there can be 0 of the coins that are heads, while also all coins being heads and any permutation inbetween, so the result would be:

		$R_X=\{0,1,2,3, \ldots, 100\}$
	
		\item I toss a coin until the first heads appears. Let Y be the total number of coin tosses.

		$R_Y=\{1,2,3,4,\ldots, \infty\}$

		\item The random variable T is defined as the time (in hours) from now until the next earthquake occurs in a certain city.

		$R_T=\{0,1,2,3, \ldots, \infty\}$
	\end{enumerate}

\subsection*{Discrete Random Variables}
X is a discrete random variable if its range is countable
\subsection*{\acrfull{pmf}}
\acrshort{pmf} is a function of the random variable X such that the notation is $P_X(x_k)$ where $k$ is the k'th element in the range of X
\subsection*{Exercise 3.3}
I toss a fair coin twice and let X be defined as the number of heads i observe. Find the rage of $X$, $R_X$, as well as its probability mass function $P_X$

$$R_X=\{0, 1, 2\}$$
$$P_X(k)=P(X=k)$$