\section*{Topic 1: Basic Concepts}
Set operations:

\begin{table}[H]
    \centering
    \begin{tabular}{|c|c|c|}
        \hline
        name & notation & description \\\hline
        union & $A\cup B$ & A combination of set A and set B \\\hline
        intersection & $A \cap B$ & the shared elements between set A and set B \\\hline
        subtraction & $A - B$ & everything in A that is not in B \\\hline
        compliment & $\neg A = A^c$ & everything that is part of S and not part of A \\\hline
        multiplication & $A X B$ & 
        
        \begin{tabular}{c}
            example: you have set $A = \{1,2,3\}$ and set $B = \{2,4,6\}$, then $A X B$\\ would be equal to $\{(1,2), (1,4), (1,6), (2,2), (2,4), (2,6), (3,2), (3,4), (3,6)\}$
        \end{tabular}\\\hline
    \end{tabular}
\end{table}

\subsection*{mutual exclusion}
If two sets have only unique elements, for set $S = \{1,2,3,4,5,6\}$, set A = {1,2,3} and set B = {4,5,6} will be mutually exclusive(disjoint)

\subsection*{Propositions}
They are usually written by the notation $P(A\cup B)$ for the set $A\cup B$

For the set S, $P(S)$ will always be 1

\subsection*{Exercises}