\section*{Lecture 1: System overview and Geometric representation of signals }

It sounds like this course will focus a lot on the low layers of the OSI-model, something like converting signals (a series of bits) to a continuous signal and back

In a communication system it is not possible to always get the same output at a receiver as the input.

\paragraph{Exercise: Annotate properties}
\begin{itemize}
    \item Information source: may be analog or digital

    The source of information could be digital, meaning a sequence of bits (data), or analog meaning a wave, maybe a speech signal

    \item Source encoder: may perform sampling, quantization and compression; generates one binary symbol (bit) $U_k \in \{0, 1\}$ per time interval $T_b$

    It will do anything to the information source to prepare it for transmission, sampling the signal if its analog and supposed to travel over a digital medium, it can be quantized to reduce size while retaining a resemblance to the original, or compressed meaning it can be reduced in size without losing data. It generates one bit per time interval $T_b$

    \item Waveform modulator: generates one waveform x(t) per time interval T

    generates the signal to be transmitted as a continuous function

    \item Vector decoder: generates one binary symbol (bit) $\hat U_k \in \{0, 1\}$ (estimates of $U_k$) per time interval $T_b$

    it takes the continuous signal from the waveform modulator and decodes it into a bit for every time interval $T_b$

    \item Source decoder: reconstructs the original source signal

    takes the bits from the vector decoder and reconstructs it to what was desired to send to the receiver.
\end{itemize}